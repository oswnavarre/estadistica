\documentclass[]{book}
\usepackage{lmodern}
\usepackage{amssymb,amsmath}
\usepackage{ifxetex,ifluatex}
\usepackage{fixltx2e} % provides \textsubscript
\ifnum 0\ifxetex 1\fi\ifluatex 1\fi=0 % if pdftex
  \usepackage[T1]{fontenc}
  \usepackage[utf8]{inputenc}
\else % if luatex or xelatex
  \ifxetex
    \usepackage{mathspec}
  \else
    \usepackage{fontspec}
  \fi
  \defaultfontfeatures{Ligatures=TeX,Scale=MatchLowercase}
\fi
% use upquote if available, for straight quotes in verbatim environments
\IfFileExists{upquote.sty}{\usepackage{upquote}}{}
% use microtype if available
\IfFileExists{microtype.sty}{%
\usepackage{microtype}
\UseMicrotypeSet[protrusion]{basicmath} % disable protrusion for tt fonts
}{}
\usepackage[margin=1in]{geometry}
\usepackage{hyperref}
\hypersetup{unicode=true,
            pdftitle={Análisis Estadístico de Datos Financieros con R},
            pdfauthor={Oswaldo Navarrete Carreño, María Alexandra Chávez},
            pdfborder={0 0 0},
            breaklinks=true}
\urlstyle{same}  % don't use monospace font for urls
\usepackage{natbib}
\bibliographystyle{apalike}
\usepackage{color}
\usepackage{fancyvrb}
\newcommand{\VerbBar}{|}
\newcommand{\VERB}{\Verb[commandchars=\\\{\}]}
\DefineVerbatimEnvironment{Highlighting}{Verbatim}{commandchars=\\\{\}}
% Add ',fontsize=\small' for more characters per line
\usepackage{framed}
\definecolor{shadecolor}{RGB}{248,248,248}
\newenvironment{Shaded}{\begin{snugshade}}{\end{snugshade}}
\newcommand{\KeywordTok}[1]{\textcolor[rgb]{0.13,0.29,0.53}{\textbf{#1}}}
\newcommand{\DataTypeTok}[1]{\textcolor[rgb]{0.13,0.29,0.53}{#1}}
\newcommand{\DecValTok}[1]{\textcolor[rgb]{0.00,0.00,0.81}{#1}}
\newcommand{\BaseNTok}[1]{\textcolor[rgb]{0.00,0.00,0.81}{#1}}
\newcommand{\FloatTok}[1]{\textcolor[rgb]{0.00,0.00,0.81}{#1}}
\newcommand{\ConstantTok}[1]{\textcolor[rgb]{0.00,0.00,0.00}{#1}}
\newcommand{\CharTok}[1]{\textcolor[rgb]{0.31,0.60,0.02}{#1}}
\newcommand{\SpecialCharTok}[1]{\textcolor[rgb]{0.00,0.00,0.00}{#1}}
\newcommand{\StringTok}[1]{\textcolor[rgb]{0.31,0.60,0.02}{#1}}
\newcommand{\VerbatimStringTok}[1]{\textcolor[rgb]{0.31,0.60,0.02}{#1}}
\newcommand{\SpecialStringTok}[1]{\textcolor[rgb]{0.31,0.60,0.02}{#1}}
\newcommand{\ImportTok}[1]{#1}
\newcommand{\CommentTok}[1]{\textcolor[rgb]{0.56,0.35,0.01}{\textit{#1}}}
\newcommand{\DocumentationTok}[1]{\textcolor[rgb]{0.56,0.35,0.01}{\textbf{\textit{#1}}}}
\newcommand{\AnnotationTok}[1]{\textcolor[rgb]{0.56,0.35,0.01}{\textbf{\textit{#1}}}}
\newcommand{\CommentVarTok}[1]{\textcolor[rgb]{0.56,0.35,0.01}{\textbf{\textit{#1}}}}
\newcommand{\OtherTok}[1]{\textcolor[rgb]{0.56,0.35,0.01}{#1}}
\newcommand{\FunctionTok}[1]{\textcolor[rgb]{0.00,0.00,0.00}{#1}}
\newcommand{\VariableTok}[1]{\textcolor[rgb]{0.00,0.00,0.00}{#1}}
\newcommand{\ControlFlowTok}[1]{\textcolor[rgb]{0.13,0.29,0.53}{\textbf{#1}}}
\newcommand{\OperatorTok}[1]{\textcolor[rgb]{0.81,0.36,0.00}{\textbf{#1}}}
\newcommand{\BuiltInTok}[1]{#1}
\newcommand{\ExtensionTok}[1]{#1}
\newcommand{\PreprocessorTok}[1]{\textcolor[rgb]{0.56,0.35,0.01}{\textit{#1}}}
\newcommand{\AttributeTok}[1]{\textcolor[rgb]{0.77,0.63,0.00}{#1}}
\newcommand{\RegionMarkerTok}[1]{#1}
\newcommand{\InformationTok}[1]{\textcolor[rgb]{0.56,0.35,0.01}{\textbf{\textit{#1}}}}
\newcommand{\WarningTok}[1]{\textcolor[rgb]{0.56,0.35,0.01}{\textbf{\textit{#1}}}}
\newcommand{\AlertTok}[1]{\textcolor[rgb]{0.94,0.16,0.16}{#1}}
\newcommand{\ErrorTok}[1]{\textcolor[rgb]{0.64,0.00,0.00}{\textbf{#1}}}
\newcommand{\NormalTok}[1]{#1}
\usepackage{longtable,booktabs}
\usepackage{graphicx,grffile}
\makeatletter
\def\maxwidth{\ifdim\Gin@nat@width>\linewidth\linewidth\else\Gin@nat@width\fi}
\def\maxheight{\ifdim\Gin@nat@height>\textheight\textheight\else\Gin@nat@height\fi}
\makeatother
% Scale images if necessary, so that they will not overflow the page
% margins by default, and it is still possible to overwrite the defaults
% using explicit options in \includegraphics[width, height, ...]{}
\setkeys{Gin}{width=\maxwidth,height=\maxheight,keepaspectratio}
\IfFileExists{parskip.sty}{%
\usepackage{parskip}
}{% else
\setlength{\parindent}{0pt}
\setlength{\parskip}{6pt plus 2pt minus 1pt}
}
\setlength{\emergencystretch}{3em}  % prevent overfull lines
\providecommand{\tightlist}{%
  \setlength{\itemsep}{0pt}\setlength{\parskip}{0pt}}
\setcounter{secnumdepth}{5}
% Redefines (sub)paragraphs to behave more like sections
\ifx\paragraph\undefined\else
\let\oldparagraph\paragraph
\renewcommand{\paragraph}[1]{\oldparagraph{#1}\mbox{}}
\fi
\ifx\subparagraph\undefined\else
\let\oldsubparagraph\subparagraph
\renewcommand{\subparagraph}[1]{\oldsubparagraph{#1}\mbox{}}
\fi

%%% Use protect on footnotes to avoid problems with footnotes in titles
\let\rmarkdownfootnote\footnote%
\def\footnote{\protect\rmarkdownfootnote}

%%% Change title format to be more compact
\usepackage{titling}

% Create subtitle command for use in maketitle
\newcommand{\subtitle}[1]{
  \posttitle{
    \begin{center}\large#1\end{center}
    }
}

\setlength{\droptitle}{-2em}

  \title{Análisis Estadístico de Datos Financieros con R}
    \pretitle{\vspace{\droptitle}\centering\huge}
  \posttitle{\par}
    \author{Oswaldo Navarrete Carreño, María Alexandra Chávez}
    \preauthor{\centering\large\emph}
  \postauthor{\par}
      \predate{\centering\large\emph}
  \postdate{\par}
    \date{A quién aún no ha visto la luz y ya ilumina mi vida.}

\ifxetex
  \usepackage{polyglossia}
  \setmainlanguage{spanish}
  % Tabla en lugar de cuadro
  \gappto\captionsspanish{\renewcommand{\tablename}{Tabla}  
          \renewcommand{\listtablename}{Índice de tablas}}
\else
  \usepackage[spanish,es-tabla]{babel}
\fi

\begin{document}
\maketitle

{
\setcounter{tocdepth}{1}
\tableofcontents
}
\chapter{¿A quién va dirigido este
libro?}\label{a-quien-va-dirigido-este-libro}

Este libro no es una introducción a la estadística. En la presente obra
se intenta hacer un repaso de algunos temas de estadistica que debe
conocer quien desee hacer investigación en Contabilidad, en Auditoría o
quizás en alguna ciencia social. Es probable que se omitan algunas cosas
pero la retroalimentación de los lectores de esta obra será importante
para su crecimiento.

En este texto se presentan, discuten y aplican los conceptos. La
presentación de los conceptos es realizada pensando en un diálogo entre
el autor y el lector, sin descuidar la formalidad de las expresiones
matemáticas. Para la discusión y aplicación de los conceptos, se va
mostrando al usuario como implementar el análisis estadístico en R.

Para aprovechar al máximo este libro se recomienda tener a mano una
computadora con R instalado, a fin de poder ir ejecutando los códigos
que se muestran. Los scripts y los conjuntos de datos que se presentan
pueden ser descargados de \url{https://github.com/oswnavarre/AEDFCR}

Aunque la obra tiene un enfoque práctico, el lector no debe olvidar que
aprender a usar R no implica saber estadística y que los programas
estadísticos no brindan soluciones si el usuario no conoce los conceptos
que deben ser aplicados.

\section{Instalando R Y Rstudio}\label{instalando-r-y-rstudio}

R es un lenguaje y entorno para computación estadística y gráficos. En
los últimos años el uso del programa estadístico R ha ido en aumento.
Puede ser descargado de \url{https://cran.r-project.org/}. Una de las
características interesantes del programa es que su capacidad puede ser
incrementada con la ayuda de paquetes, en la actualidad la página
oficial del programa tiene cerca de 14000 paquetes.

RStudio es una interfaz que ayuda a explotar todas las capacidades de R.
Rstudio se descarga de la página \url{https://www.rstudio.com/}.

\chapter{Introducción}\label{intro}

En nuestra vida diaria es común escuchar el término
\textbf{estadística}, las tasas de desempleo, el índice de pobreza, el
saldo promedio de nuestra cuenta de ahorros, el número de goles
realizados en la LigaPro durante el fin de semana, etc. Aunque no es una
forma incorrecta de ver las estadísticas, en este texto se pensará a la
estadística como un conjunto de métodos que se utilizan para
\textbf{recoger, clasificar, resumir, organizar, presentar, analizar e
interpretar información numérica}

En las empresas la estadística es usada para tomar decisiones como los
productos y las cantidades que deben ser producidas, la frecuencia con
la que una maquinaria debe recibir mantenimiento, el tamaño del
inventario, la forma de distribuir los productos, y casi todos los
aspectos relativos a sus operaciones. En el estudio de las finanzas, la
contabilidad, la economía y otras ciencias sociales la motivación para
usar estadística radica en entender como funcionan los sistemas
económicos, financieros o contables.

\section{Estadística descriptiva e
inferencial}\label{estadistica-descriptiva-e-inferencial}

El uso de la estadística puede ser de dos formas. La primera, cuando se
describen y se presentan los datos. Y la segunda es cuando los datos son
utilizados para hacer inferencias sobre características del ambiente o
entorno de donde se seleccionaron los datos o sobre el mecanismo
subyacente que generó los datos. La primera forma recibe el nombre de
\textbf{estadística descriptiva} y la segunda se conoce como
\textbf{estadística inferencial}

En la estadística descriptiva se utilizan métodos numéricos y gráficos
para encontrar patrones y características de los datos a fin de resumir
la información y presentarla de una forma significativa. Mientras que en
la estadística inferencial se utilizan los daots para tomar decisiones,
hacer estimaciones, pronósticos o predicciones y generalizaciones sobre
el entorno del que fueron obtenidos los datos o el proceso que los
generó.

Sea en estadística descriptiva o en estadística inferencial, el primer
paso siempre va a ser obtener información de alguna característica,
medida o valor que nos interese de un grupo de elementos. Esa
característica, medida o valor de interés para el investigador recibe el
nombre de \textbf{variable}.

\section{Tipos de Variables}\label{tipos-de-variables}

Muchos autores presentan algunas clasificaciones para las variables, sin
embargo vamos a trabajar con una clasificación que se ajusta a las
necesidades de la investigación en las áreas de nuestro interés. Según
esta clasificación hay dos grandes grupos de variables: cuantitativas y
cualitativas. Las primeras son las que toman valores \textbf{numéricos}.
Mientras que las cualitativas toman valores que describen una
\textbf{cualidad} o \textbf{característica}.

Las variables cuantitativas se clasifican a la vez en \textbf{continuas}
que se presentan cuando las observaciones pueden tomar cualquier valor
dentro de un subconjunto de los números reales, ejemplos de variables
cuantitativas continuas son: edad, altura, temperatura y peso. Las
\textbf{discretas} son aquellas cuya característica principal es que las
observaciones pueden tomar un valor basado en un recuento de un conjunto
de valores enteros distintos. Ejemplos de variables cuantitativas
discretas son: número de hijos, número de comprobantes de venta emitidos
en un mes, número de clientes haciendo fila durante una hora en un
banco.

\subsection{Niveles de medición}\label{niveles-de-medicion}

Hay cuatro niveles de medición \textbf{ordinal}, \textbf{nominal},
\textbf{intervalo} y de \textbf{radio}. En el nivel ordinal las
observaciones toman valores que se ordenan o clasifican de forma lógica,
por ejemplo las tallas de ropa (pequeña, media, grande, extra grande),
la frecuencia con la que se hace una actividad (nunca, casi nunca, a
veces, casi siempre, siempre). Por otro lado, en el nivel nominal las
observaciones toman valores que no se pueden organizar de forna lógica,
por ejemplo el sexo, el color de ojos, la marca de ropa favorita.

En el nivel de intervalo existe diferencia significativa entre valores
pero el cero no representa la ausencia de la característica un ejemplo
es la temperatura medida en grados Farenheit. Finalmente en el nivel de
razón el 0 es significativo y la razón entre dos números es
significativa, un ejemplo es la temperatura medida en grados Kelvin.

\section{Primeros pasos en R}\label{primeros-pasos-en-r}

Una vez instalado R y RStudio, abrimos Rstudio para comenzar a trabajar.
La ventana de RStudio se ve como se muestra en la figura
\ref{fig:rstudio1}.

\begin{figure}[h]

{\centering \includegraphics[width=0.5\linewidth]{rstudio1} 

}

\caption{Ventana de RStudio}\label{fig:rstudio1}
\end{figure}

Lo primero que debemos hacer es abrir un nuevo script, un script de R es
simplemente un archivo de texto que contiene (casi) todos los comandos
que se escribirían en la línea de comandos de R, para esto en la barra
de menú seguimos la secuencia \textbf{File, New File, R Script} o desde
el teclado con la combinación \emph{Ctrl + Shift + N}, en este archivo
iremos escribiendo todos los comandos que vamos a trabajar. En la figura
\ref{fig:rstudio2} se aprecia un script abierto.

\begin{figure}[h]

{\centering \includegraphics[width=0.5\linewidth]{rstudio2} 

}

\caption{Ventana de RStudio con Script}\label{fig:rstudio2}
\end{figure}

Para empezar a aprender en el script vamos a escribir \texttt{3+2} y
ejecutamos esto con la combinación de teclas \textbf{Ctrl + Enter} el
resultado obviamente es 5. Ahora ingresaremos un conjunto de valores y
los almacenaremos en una variable, para almacenar algo en una variable
se puede usar \texttt{\textless{}-} o \texttt{=}. En la variable
\texttt{x} almacenaremos un conjunto de 8 observaciones escribiendo el
código:

\begin{Shaded}
\begin{Highlighting}[]
\NormalTok{x <-}\StringTok{ }\KeywordTok{c}\NormalTok{(}\DecValTok{3}\NormalTok{,}\DecValTok{7}\NormalTok{,}\DecValTok{9}\NormalTok{,}\DecValTok{5}\NormalTok{,}\DecValTok{6}\NormalTok{,}\DecValTok{2}\NormalTok{,}\DecValTok{1}\NormalTok{,}\DecValTok{10}\NormalTok{) }
\end{Highlighting}
\end{Shaded}

Recuerde que este código se ejecuta con la combinación de teclas
\textbf{Ctrl + Enter}. Para poder realizar análisis estadístico, es
necesario cargar nuestros datos en el programa. R acepta algunos
formatos de archivos, como por ejemplo archivos de Excel, archivos de
valores separados por coma, archivos de texto e inclusive archivos de
otros programas como SPSS. Lo más usual es trabajar con archvo de
valores separados por coma es decir con extensión \texttt{.csv}, estos
archivos \texttt{csv} se generan cuando el investigador recolecta la
información, la almacena en un archivo de Excel o alguna otra hoja de
cálculo y la guarda como un archivo de valores separados por coma.

Para trabajar de forma eficiente con R, se recomienda comenzar por fijar
un directorio de trabajo donde deben estar guardados nuestros archivos
en el formato que sea de nuestra preferencia. Una forma de hacerlo es
desde la barra de menú \textbf{Session, Set Working Directory, Choose
Directory} o desde el teclado con la combinación \textbf{Ctrl+Shift+H},
o con la función \texttt{setwd("rutadelarchivo")}.

En este primer ejercicio trabajaremos con el archivo
\emph{cap2\_big4\_size.csv}. Los datos serán guardados en una variable
llamada \texttt{datos1}, usaremos la función \texttt{read.csv()} para
leer los datos. La función \texttt{read.csv()} recibe las instrucciones
\texttt{read.csv("archivo",\ header=T,\ sep=";",dec=",")}. La opción
\texttt{"archivo"} indica el nombre del archivo, \texttt{header=T} o
\texttt{header=F} permite indicar si las columnas tienen un encabezado
que las identifique, \texttt{sep=";"} sirve para indicar cual es el
separador presente en nuestro archivo en algunas ocasiones ocurre que un
archivo de valores separado por coma en realidad tiene sus valores
separados por un punto y cona esto generalmente ocurre cuando el sistema
utiliza, como en este caso, la coma como separador decimal y finalmente
la opción \texttt{dec=","} sirve para indicar que el separador decimal
es la coma.

Una característica de R es que permite acceder a la ayuda sobre las
funciones, esto se hace escribiendo \texttt{?funcion} por ejemplo si
queremos la ayuda de la función \texttt{read.csv} simplemente escribimos
\texttt{?read.csv} en el panel ubicado en la parte inferior derecha se
desplegará la ayuda de la función. Con la particularidad de que la ayuda
se despliega en inglés lo que no debería ser problema para un buen
investigador.

El archivo que vamos a analizar contiene los activos, la utilidad, las
ventas y el patrimonio de una muestra de empresas tomada de los
registros de la Superintendencia de Compañías. Además en el conjunto de
datos se indica si la empresa ha sido auditada por una de las 4 firmas
auditoras consideradas las más grandes o también llamadas
\emph{\emph{Big Four}}. En la \ref{tab:tabla1} se muestran las 10
primeras observaciones de nuestro conjunto de datos.

\begin{table}[t]

\caption{\label{tab:tabla1}Primeras 10 observaciones}
\centering
\begin{tabular}{rrrrrr}
\toprule
EXPMUESTRA & BIG4 & ACTIVOS & UTILIDAD & VTAS & PAT\\
\midrule
85 & 1 & 73315618 & 7522758.7 & 191474544 & 39382529\\
100121 & 0 & 21052702 & -122898.5 & 132585022 & 1577764\\
45178 & 0 & 10468672 & 536876.9 & 13974269 & 4312094\\
51193 & 0 & 4130483 & 455759.4 & 8670153 & 1858990\\
47598 & 0 & 23507401 & 266370.5 & 18555609 & 7137609\\
\addlinespace
31720 & 0 & 7220312 & 437718.3 & 16097135 & 4002154\\
46189 & 0 & 14526822 & 1206400.9 & 12281188 & 5015806\\
9731 & 0 & 8539445 & 367848.1 & 10844918 & 2232339\\
4619 & 0 & 2605059 & -22438.4 & 6244589 & 1366610\\
102434 & 0 & 23975816 & 790265.6 & 40612649 & 8754369\\
\bottomrule
\end{tabular}
\end{table}

Sin más preámbulos, empecemos a trabajar. Recapitulando, primero
configuraremos el directorio de trabajo, luego cargaremos el archivo
indicado. Finalmente usamos la función \texttt{str()}, la que nos
permite obtener la descripción de la estructura de los datos.

\begin{Shaded}
\begin{Highlighting}[]
\KeywordTok{setwd}\NormalTok{(}\StringTok{"C:/Users/onava_000/OneDrive/libro_mc/estadistica"}\NormalTok{)}
\NormalTok{datos1 <-}\StringTok{ }\KeywordTok{read.csv}\NormalTok{(}\StringTok{"cap2_big4_size.csv"}\NormalTok{,}\DataTypeTok{header=}\OtherTok{TRUE}\NormalTok{,}\DataTypeTok{sep=}\StringTok{";"}\NormalTok{,}\DataTypeTok{dec=}\StringTok{","}\NormalTok{)}
\KeywordTok{str}\NormalTok{(datos1)}
\end{Highlighting}
\end{Shaded}

\begin{verbatim}
## 'data.frame':    2256 obs. of  6 variables:
##  $ EXPMUESTRA: int  85 100121 45178 51193 47598 31720 46189 9731 4619 102434 ...
##  $ BIG4      : int  1 0 0 0 0 0 0 0 0 0 ...
##  $ ACTIVOS   : num  73315618 21052702 10468672 4130483 23507401 ...
##  $ UTILIDAD  : num  7522759 -122898 536877 455759 266371 ...
##  $ VTAS      : num  1.91e+08 1.33e+08 1.40e+07 8.67e+06 1.86e+07 ...
##  $ PAT       : num  39382529 1577764 4312094 1858990 7137609 ...
\end{verbatim}

En la primera línea de los resultados se observa la salida
\texttt{\textquotesingle{}data.frame\textquotesingle{}:\ \ \ \ 2256\ obs.\ of\ \ 6\ variables:}
esto nos indica que nuestro \emph{marco de datos (data frame)} tiene
2256 observaciones y 6 variables. Con respecto a las variables tenemos 6
variables que a continuación se describen y se explican los resultados
obtenidos con la función.

\begin{itemize}
\tightlist
\item
  \texttt{EXPMUESTRA}: esta variable es de tipo entera (INT) y almacena
  el expediente de la empresa. Aunque la variable tiene valores
  numéricos, no es una variable cuantitativa sino cualitativa
  ``Expediente de la Empresa''
\item
  \texttt{BIG4}: esta variable es de tipo entera, y ha sido codificada
  con 1 si la empresa fue auditada por una Big Four y 0 si no. Podemos
  cambiar esta codificación por ``Sí'' y ``No'' en lugar de ``1'' y
  ``0'', más adelante aprendemos como hacerlo. Al igual que la variable
  anterior aunque tiene valores numéricos, no es una variable
  cuantitativa sino cualitativa, dejamos al lector la reflexión en este
  particular.
\item
  \texttt{ACTIVOS}: contiene el valor de los activos totales de la
  empresa. Es de tipo \texttt{num} porque permite el uso de decimales.
  Corresponde a una variable cuantitativa continua.
\item
  \texttt{UTILIDAD}: contiene el valor de la utildad de la empresa.
\item
  \texttt{VTAS}: contiene el valor de las ventas de la empresa.
\item
  \texttt{PAT}: contiene el valor del patrimonio de la empresa.
\end{itemize}

Los paquetes de R son colecciones de funciones y conjuntos de datos
desarrollados por la comunidad de usuarios, los paquetes aumentan el
poder de R mejorando las funcionalidades existentes en la base de R, o
añadiendo nuevas funcionalidades. En este texto trabajaremos con algunos
de los paquetes desarrollados por el equipo de RStudio, una descripción
detallada de estos paquetes puede ser encontrada en
\url{https://www.rstudio.com/products/rpackages/}. . Comenzaremos por
instalar el paquete \texttt{dplyr}, este paquete tiene funciones que
permiten realizar facilmente manipulaciones de datos. Para instalar un
paquete se utiliza la función \texttt{install.packages("paquete")}. Una
vez instalado el paquete, se carga el paquete utilizando la función
\texttt{library(paquete)}.

\begin{Shaded}
\begin{Highlighting}[]
\KeywordTok{install.packages}\NormalTok{(}\StringTok{"dplyr"}\NormalTok{)}
\end{Highlighting}
\end{Shaded}

La primera manipulación que vamos a realizar es la creación de nuevas
variables con el paquete \texttt{dplyr}. En nuestros datos cargados en
el conjunto de datos \texttt{datos1} vamos a crear tres variables nuevas
\textbf{ROA}, \textbf{ROS} y \textbf{ROE}. Recordemos que el
\textbf{Retorno sobre activos} ( \textbf{ROA}, Return on Assets) se lo
calcula como la razón entre la utilidad y los activos como se ve en la
ecuación \eqref{eq:roa}. En las ecuaciones \eqref{eq:ros} y \eqref{eq:roe} se
dan las expresiones para calcular el \textbf{Retorno sobre ventas} (
\textbf{ROS} Return on Sales) y el \textbf{Retorno sobre el Patrimonio}
( \textbf{ROE} Return on Equity)

\begin{equation} 
  ROA = \dfrac{Utilidad}{Activos}
  \label{eq:roa}
\end{equation}

\begin{equation} 
  ROS = \dfrac{Utilidad}{Ventas}
  \label{eq:ros}
\end{equation}

\begin{equation} 
  ROE = \dfrac{Utilidad}{Patrimonio}
  \label{eq:roe}
\end{equation}

Una característica importante de \texttt{dplyr} es el uso del operador
\texttt{\%\textgreater{}\%}. Cada transformación u operación en los
datos se separa por el operador \texttt{\%\textgreater{}\%}. La primera
función de \texttt{dplyr} que usaremos es \texttt{mutate()}, básicamente
esta función permite crear nuevas variables.

\begin{Shaded}
\begin{Highlighting}[]
\KeywordTok{library}\NormalTok{(dplyr)}
\NormalTok{datos1 <-}\StringTok{ }\NormalTok{datos1 }\OperatorTok
\StringTok{  }\KeywordTok{mutate}\NormalTok{(}
    \DataTypeTok{ROA =}\NormalTok{ UTILIDAD}\OperatorTok{/}\NormalTok{ACTIVOS,}
    \DataTypeTok{ROS =}\NormalTok{ UTILIDAD}\OperatorTok{/}\NormalTok{VTAS,}
    \DataTypeTok{ROE =}\NormalTok{ UTILIDAD}\OperatorTok{/}\NormalTok{PAT}
\NormalTok{  )}
\KeywordTok{str}\NormalTok{(datos1)}
\end{Highlighting}
\end{Shaded}

\begin{verbatim}
## 'data.frame':    2256 obs. of  9 variables:
##  $ EXPMUESTRA: int  85 100121 45178 51193 47598 31720 46189 9731 4619 102434 ...
##  $ BIG4      : int  1 0 0 0 0 0 0 0 0 0 ...
##  $ ACTIVOS   : num  73315618 21052702 10468672 4130483 23507401 ...
##  $ UTILIDAD  : num  7522759 -122898 536877 455759 266371 ...
##  $ VTAS      : num  1.91e+08 1.33e+08 1.40e+07 8.67e+06 1.86e+07 ...
##  $ PAT       : num  39382529 1577764 4312094 1858990 7137609 ...
##  $ ROA       : num  0.10261 -0.00584 0.05128 0.11034 0.01133 ...
##  $ ROS       : num  0.039289 -0.000927 0.038419 0.052566 0.014355 ...
##  $ ROE       : num  0.191 -0.0779 0.1245 0.2452 0.0373 ...
\end{verbatim}

En las últimas líneas de la salida de R, se observa que ahora en el
conjunto de datos existen ahora tres nuevas variables. En la próxima
sección seguiremos trabajando con el mismo conjunto de datos.

\section{Medidas de Tendencia
Central}\label{medidas-de-tendencia-central}

Una medida de tendencia central, es una medida de resumen que intenta
describir un conjunto completo de datos con un único valor que
representa la mitad o centro de la distribución.

Las tres medidas de tendencia central principales son la media la
mediana y la moda.

\subsection{Media}\label{media}

La media se la calcula como la suma de todos los valores de una variable
dividido para el número de valores. En la ecuación \eqref{eq:mean} se
muestra la fórmula para calcular la media.

\begin{equation} 
  \bar{x} = \dfrac{\sum_{i=1}^{n}x_i}{n}
  \label{eq:mean}
\end{equation}

La media tiene algunas propiedades que a continuación se detallan:

\begin{itemize}
\tightlist
\item
  Si a cada valor \(x_i\) de una distribución con media \(\bar{x}\) se
  le suma un valor constante \(k \in \mathbb{R}\), la nueva media es
  \(\bar{x}+k\)
\item
  Si a cada valor \(x_i\) de una distribución con media \(\bar{x}\) se
  lo multiplica por un valor constante \(k \in \mathbb{R}\), la nueva
  media es \(k\bar{x}\)
\item
  Si a cada valor \(x_i\) de una distribución con media \(\bar{x}\) se
  lo divide por un valor constante \(k \neq 0 \in \mathbb{R}\), la nueva
  media es \(\frac{\bar{x}}{k}\)
\end{itemize}

Las ventajas de usar la media son:

\begin{itemize}
\tightlist
\item
  Es fácil de entender y calcular
\item
  No se ve afectada mayormente por fluctuaciones productos del muestreo
\item
  Toma en cuenta todos los valores de la variable
\end{itemize}

Las desventajas de usar la media son:

\begin{itemize}
\tightlist
\item
  Es muy sensible a la presencia de pocos valores muy pequeños o muy
  grandes, dicho de otra forma la media es sensible a valores
  aberrantes.
\item
  No se puede calcular por inspección.
\end{itemize}

\subsection{Mediana}\label{mediana}

La mediana es el valor central en una distribución cuando se ordenan los
valores de forma ascendente o descendente. El valor de la mediana
depende entonces del número de valores presentes en la variable.
Definamos como \(\left\{ X \right \}\) al conjunto de datos ordenado, y
sea \(\left \{ X \right \}_i\) el valor i-ésimo del conjunto
\(\left \{ X \right \}\) entonces la mediana \(Me\) se define como

\begin{equation}
Me = \begin{cases} 
      \left \{ X \right\}_{\frac{n+1}{2}} & ; n \quad \textrm{impar}  \\
      \dfrac{\left \{ X  \right \}_{\frac{n}{2}} + \left \{ X  \right \}_{\frac{n}{2}+1} }{2} & ; n \quad \textrm{par}
   \end{cases}
   \label{eq:median}
\end{equation}

Lo escrito en la ecuación \eqref{eq:median} se puede expresar de la
siguiente forma: si el número de datos es impar, la mediana es igual al
valor central de la distribución y si el número de datos es par, la
mediana es igual al promedio de los valores centrales de la
distribución.

Las ventajas de usar la mediana son:

\begin{itemize}
\tightlist
\item
  Es fácil de calcular y comprender
\item
  No se ve afectada por valores extremos
\item
  Se puede determinar para escalas ordinales, nominales, de razón e
  intervalo
\end{itemize}

Las desventajas de usar la mediana son:

\begin{itemize}
\tightlist
\item
  No toma en cuenta el valor exacto de cada dato y por tanto no usa toda
  la información disponible.
\item
  Si se agrupan los valores de dos grupos, la mediana de cada grupo no
  puede ser expresada en términos del grupo agrupado.
\end{itemize}

\subsection{Moda}\label{moda}

La moda es definida como el valor que ocurre con mayor frecuencia en los
datos. Algunos conjuntos de datos no tienen moda porque cada valor
ocurre solo una vez. Hay conjuntos de datos que tienen más de una moda,
si tienen 2 modas reciben el nombre de bimodal y se acostumbra que si
tiene más de 3 modas se la llama multimodal.

Las ventajas de usar la moda son:

\begin{itemize}
\tightlist
\item
  Puede ser usada para datos con escala nominal
\item
  Es sencilla de calcular
\end{itemize}

La desventaja de la moda es:

\begin{itemize}
\tightlist
\item
  No es usada en análisis estadístico debido a que no está definida
  algebraicamente y la fluctuación en la frecuencia de las observaciones
  es mayor cuando el tamaño de la muestra es pequeña.
\end{itemize}

\subsection{¿trabajamos con la media o la
mediana?}\label{trabajamos-con-la-media-o-la-mediana}

La media es considerada generalmente la mejor medida de tendencia
central y la más usada. Sin embargo, hay situaciones donde las otras
medidas de tendencia central son preferidas.

La mediana es preferida a la media cuando:

\begin{itemize}
\tightlist
\item
  Hay valores extremos en la distribución
\item
  Hay valores indeterminados
\item
  Los datos son medidos en una escala ordinal
\end{itemize}

La moda es la medida preferida cuando los datos son medidos en una
escala nominal.

\subsection{Cálculo de las medidas de tendencia central en
R}\label{calculo-de-las-medidas-de-tendencia-central-en-r}

Para calcular la media y la mediana se utilizan las funciones
\texttt{mean()} y \texttt{median()} respectivamente, estas dos funciones
vienen cargadas con los paquetes base de R. Para calcular la moda
usaremos la función \texttt{Mode()} del paquete \texttt{DescTools},
recuerde que para instalar un paquete se utiliza la función
\texttt{install.packages()}.

En el siguiente ejemplo se obtiene la media de los activos de las
empresas. como solamente necesitamos una variable del conjunto de datos
usamos el operador \texttt{\$}, el funcionamiento de este operador es
\texttt{data.frame\$variable} es decir indicamos el conjunto de datos
del que llamamos la variable y después del operador \texttt{\$}
indicamos la variable que vamos a trabajar.

\begin{Shaded}
\begin{Highlighting}[]
\KeywordTok{mean}\NormalTok{(datos1}\OperatorTok{$}\NormalTok{ACTIVOS)}
\end{Highlighting}
\end{Shaded}

\begin{verbatim}
## [1] 44064165
\end{verbatim}

\begin{Shaded}
\begin{Highlighting}[]
\KeywordTok{median}\NormalTok{(datos1}\OperatorTok{$}\NormalTok{ACTIVOS)}
\end{Highlighting}
\end{Shaded}

\begin{verbatim}
## [1] 10326361
\end{verbatim}

\begin{Shaded}
\begin{Highlighting}[]
\KeywordTok{library}\NormalTok{(DescTools)}
\KeywordTok{Mode}\NormalTok{(datos1}\OperatorTok{$}\NormalTok{ACTIVOS)}
\end{Highlighting}
\end{Shaded}

\begin{verbatim}
## [1]  55996406 628446149
\end{verbatim}

En el resultado de la moda se obtienen 2 valores. Es decir que existen
dos valores que se repiten más veces o tienen mayor frecuencia. Cuando
se realiza investigación es común desear hacer una tabla con las
estadísticas descriptivas de los datos. El paquete \texttt{dplyr}
permite realizar tablas que resuman las variables de forma sencilla

\begin{Shaded}
\begin{Highlighting}[]
\NormalTok{datos1 }\OperatorTok
\StringTok{  }\KeywordTok{summarise}\NormalTok{(}\DataTypeTok{PROM.ACTIVOS =} \KeywordTok{mean}\NormalTok{(ACTIVOS),}
            \DataTypeTok{PROM.UTILIDAD =} \KeywordTok{mean}\NormalTok{(UTILIDAD),}
            \DataTypeTok{PROM.VTAS =} \KeywordTok{mean}\NormalTok{(VTAS),}
            \DataTypeTok{MEDIAN.ACTIVOS =} \KeywordTok{median}\NormalTok{(ACTIVOS),}
            \DataTypeTok{MEDIAN.UTILIDAD =} \KeywordTok{median}\NormalTok{(UTILIDAD),}
            \DataTypeTok{MEDIAN.VTAS =} \KeywordTok{median}\NormalTok{(VTAS)}
\NormalTok{            )}
\end{Highlighting}
\end{Shaded}

\begin{verbatim}
##   PROM.ACTIVOS PROM.UTILIDAD PROM.VTAS MEDIAN.ACTIVOS MEDIAN.UTILIDAD
## 1     44064165       4250664  50555030       10326361        350642.1
##   MEDIAN.VTAS
## 1     9190661
\end{verbatim}

\begin{Shaded}
\begin{Highlighting}[]
\KeywordTok{head}\NormalTok{(datos1)}
\end{Highlighting}
\end{Shaded}

\begin{verbatim}
##   EXPMUESTRA BIG4  ACTIVOS  UTILIDAD      VTAS      PAT          ROA
## 1         85    1 73315618 7522758.7 191474544 39382529  0.102607862
## 2     100121    0 21052702 -122898.5 132585022  1577764 -0.005837658
## 3      45178    0 10468672  536876.9  13974269  4312094  0.051284143
## 4      51193    0  4130483  455759.4   8670153  1858990  0.110340443
## 5      47598    0 23507401  266370.5  18555609  7137609  0.011331346
## 6      31720    0  7220312  437718.3  16097135  4002154  0.060623176
##             ROS         ROE
## 1  0.0392885581  0.19101766
## 2 -0.0009269409 -0.07789408
## 3  0.0384189612  0.12450492
## 4  0.0525664720  0.24516509
## 5  0.0143552550  0.03731929
## 6  0.0271923089  0.10937068
\end{verbatim}

\section{Tablas de frecuencia}\label{tablas-de-frecuencia}

\section{Tablas de Contingencia}\label{tablas-de-contingencia}

\section{Gráficos y Visualización}\label{graficos-y-visualizacion}

\subsection{Diagramas de Caja y valores
atípicos}\label{diagramas-de-caja-y-valores-atipicos}

\subsubsection{Intervalos de Confianza}\label{intervalos-de-confianza}

\section{Medidas de dispersión}\label{medidas-de-dispersion}

\begin{Shaded}
\begin{Highlighting}[]
\KeywordTok{par}\NormalTok{(}\DataTypeTok{mar =} \KeywordTok{c}\NormalTok{(}\DecValTok{4}\NormalTok{, }\DecValTok{4}\NormalTok{, .}\DecValTok{1}\NormalTok{, .}\DecValTok{1}\NormalTok{))}
\KeywordTok{plot}\NormalTok{(pressure, }\DataTypeTok{type =} \StringTok{'b'}\NormalTok{, }\DataTypeTok{pch =} \DecValTok{19}\NormalTok{)}
\end{Highlighting}
\end{Shaded}

\begin{figure}[h]

{\centering \includegraphics[width=0.8\linewidth]{estadistica_files/figure-latex/nice-fig-1} 

}

\caption{Here is a nice figure!}\label{fig:nice-fig}
\end{figure}

Reference a figure by its code chunk label with the \texttt{fig:}
prefix, e.g., see Figure \ref{fig:nice-fig}. Similarly, you can
reference tables generated from \texttt{knitr::kable()}, e.g., see Table
\ref{tab:nice-tab}.

\begin{Shaded}
\begin{Highlighting}[]
\NormalTok{knitr}\OperatorTok{::}\KeywordTok{kable}\NormalTok{(}
  \KeywordTok{head}\NormalTok{(iris, }\DecValTok{20}\NormalTok{), }\DataTypeTok{caption =} \StringTok{'Here is a nice table!'}\NormalTok{,}
  \DataTypeTok{booktabs =} \OtherTok{TRUE}
\NormalTok{)}
\end{Highlighting}
\end{Shaded}

\begin{table}[t]

\caption{\label{tab:nice-tab}Here is a nice table!}
\centering
\begin{tabular}{rrrrl}
\toprule
Sepal.Length & Sepal.Width & Petal.Length & Petal.Width & Species\\
\midrule
5.1 & 3.5 & 1.4 & 0.2 & setosa\\
4.9 & 3.0 & 1.4 & 0.2 & setosa\\
4.7 & 3.2 & 1.3 & 0.2 & setosa\\
4.6 & 3.1 & 1.5 & 0.2 & setosa\\
5.0 & 3.6 & 1.4 & 0.2 & setosa\\
\addlinespace
5.4 & 3.9 & 1.7 & 0.4 & setosa\\
4.6 & 3.4 & 1.4 & 0.3 & setosa\\
5.0 & 3.4 & 1.5 & 0.2 & setosa\\
4.4 & 2.9 & 1.4 & 0.2 & setosa\\
4.9 & 3.1 & 1.5 & 0.1 & setosa\\
\addlinespace
5.4 & 3.7 & 1.5 & 0.2 & setosa\\
4.8 & 3.4 & 1.6 & 0.2 & setosa\\
4.8 & 3.0 & 1.4 & 0.1 & setosa\\
4.3 & 3.0 & 1.1 & 0.1 & setosa\\
5.8 & 4.0 & 1.2 & 0.2 & setosa\\
\addlinespace
5.7 & 4.4 & 1.5 & 0.4 & setosa\\
5.4 & 3.9 & 1.3 & 0.4 & setosa\\
5.1 & 3.5 & 1.4 & 0.3 & setosa\\
5.7 & 3.8 & 1.7 & 0.3 & setosa\\
5.1 & 3.8 & 1.5 & 0.3 & setosa\\
\bottomrule
\end{tabular}
\end{table}

You can write citations, too. For example, we are using the
\textbf{bookdown} package \citep{R-bookdown} in this sample book, which
was built on top of R Markdown and \textbf{knitr} \citep{xie2015}.

\chapter{Pruebas de Hipótesis}\label{pruebas-de-hipotesis}

Here is a review of existing methods.

\chapter{Regresión}\label{methods}

We describe our methods in this chapter.

\chapter{Análisis Factorial}\label{analisis-factorial}

\section{Análisis de Fiabilidad}\label{analisis-de-fiabilidad}

\section{Evaluación de Análisis
Factorial}\label{evaluacion-de-analisis-factorial}

\chapter{Algo de series de tiempo}\label{algo-de-series-de-tiempo}

We have finished a nice book.

\bibliography{book.bib,packages.bib}


\end{document}
